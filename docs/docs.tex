\documentclass{article}

\usepackage[margin=1in]{geometry}

\title{Program 1 - Perceptron} \author{Logan Grosz} \date{Last updated: \today}
\setcounter{tocdepth}{2}

\setlength{\parindent}{0pt} \setlength{\parskip}{1em}

\begin{document}

\maketitle

\tableofcontents

\newpage

\section{Usage}

\subsection{Installing the module from source}

In the directory with \texttt{setup.py}

\texttt{pip install .}

\subsection{Usage}

Then anywhere...

\texttt{import <classifier> from myml}

\texttt{import myml.util as util}

For examples see \texttt{examples/}

\section{API Definitions}

\subsection{util}

\texttt{plot\_decision\_regions(X, y, classifier, resolution)} :: Will print a
2D scattery plot along with a line that separates two regions in accordance with
the perceptron passed in.

\texttt{plot\_decision\_regions\_3d(X, y, classifier, resolution)} :: Will print
a 3D scattery plot along with a plane that separates two regions in accordance
with the perceptron passed in.

\texttt{plot\_linear\_regression} :: Will print a 2D scattery plot along with a
plane that separates two regions in accordance with the perceptron passed in.
For any higher dimensions it will not print. (For final portfolio, this will do
a 3D plot as well)

\subsection{Perceptron}

\subsubsection{Description}

The perceptron follows the data fitting algorithm presented on Wikipedia
closely. It initializes weights to 0. For each sample, it calculates the actual
output,and then changes the weight to reflect the difference between this output
and the desired output. This process repeats until the specified number of
iterations is reached or there is an iteration error rate of 0. In order to
preserve data, both the number of errors and weight at each step is recorded.
These can be accessed through \textit{Perceptron}\texttt{.weights} and
\textit{Perceptron}\texttt{.errors}.

\subsubsection{Usage}

\begin{verbatim} pn = Perceptron(rate, epochs) pn.fit(X, y) pn.predict(Z)
\end{verbatim}

\texttt{X} is a numpy.ndarray of shape ($samples$, $features$)

\texttt{y} is a numpy.ndarray of shape ($samples$)

\subsubsection{Attributes}

\textit{Perceptron}\texttt{.rate} :: The learning rate of the classifier

\textit{Perceptron}\texttt{.epochs} :: The number of epochs to run when fitting.
These are not guaranteed to run if the classifier can be fit with no error at
any point.

\textit{Perceptron}\texttt{.errors} :: Errors at each epoch during fitting. This
is a numpy.ndarray with shape ($epochs$, $features$).

\textit{Perceptron}\texttt{.weights} :: Weights at each epoch during fitting.
This is a numpy.ndarray with shape ($epochs$, $features$).

\subsubsection{Methods}

\textit{Perceptron}\texttt{.\_\_init\_\_} :: The initilizer takes 2 arguments:
the first is the learning rate, a float between 0 and 1. The second is the max
number of iterations. These values can be accessed again with
\textit{Perceptron}\texttt{.rate} and \textit{Perceptron}\texttt{.niter}.

\textit{Perceptron}\texttt{.fit} :: Takes arguments \texttt{X} (numpy array with
shape (\textit{nSamples}, \textit{nFeatures})) and \texttt{d} (also a numpy
array with shape (\textit{nSamples})). This function applies the steps described
above and ``validates" the errors and weights arrays.

\textit{Perceptron}\texttt{.net\_input} :: Returns the weighted values of some
given input \texttt{X} (of type numpy array with shape (\textit{nSamples,
nFeatures})) as a numpy array with shape (\textit{nSamples}).

\textit{Perceptron}\texttt{.predict} :: Returns labels (-1 or 1) for a given
\texttt{X} of type numpy array with shape (\textit{nSamples, nFeatures}) as a
numpy array of shape (\textit{nSamples}).

\texttt{Perceptron.f} :: a static method which returns the label given
\texttt{w} (a numpy array of shape (\textit{nFeatures}+1)), and \texttt{x}, a
sample with shape (\textit{nFeatures}). Note, they're not the same size but
that's because \texttt{w[0]} acts as a bias.

\subsection{LinearRegressor}

\subsubsection{Description}

The linear regressor fits data assumed to be linear in nature. It uses a
gradient descent algorithm on the squared error between expected and actual
output values. This can be done with any number of independant variables. It
starts assuming the weights, $\beta = [0, ..., 0]$.  It then calulates partial
derivatives of the error term $J$, for each independant variable in $X$.  $J$,
the cost, is defined as the sum of the squared error for each sample point:
$\sum_{i=1}^{n}(y_i-(\mathbf{\beta}\cdot\mathbf{X^T_i}))^2$.  Using the partial
derivatives and the error rate, a new $\beta$ is determined for the next epoch.
This is repeated until the given number of epochs.

\subsubsection{Usage}

\begin{verbatim} lr = LinearRegressor(rate, epochs) lr.fit(X, y) lr.predict(Z)
\end{verbatim}

\texttt{X} is a numpy.ndarray of shape ($n$, $p$)

\texttt{y} is a numpy.ndarray of shape ($n$)

where $n$ is the number of samples and $p$ is the number of independant
variables

\subsubsection{Properties}

Note: Attributes are meant to be read-only. Modify them at your own risk. I have
not chosen to enforce this since anyone can break it if they wanted anyways.

\textit{LinearRegressor}\texttt{.rate} :: learning rate

\textit{LinearRegressor}\texttt{.epochs} :: epochs

\textit{LinearRegressor}\texttt{.w} :: matrix of values of $\beta$. This is a
numpy.ndarray of shape ($epochs$, $p$).

\textit{LinearRegressor}\texttt{.errors} :: matrix of values of $\epsilon$.
$\epsilon$ is the error between each predicted $y_i$ and its real value, $d_i$,
such that $\epsilon + y_i = d_i$

\subsubsection{Methods}

\textit{LinearRegressor}\texttt{.\_\_init\_\_(rate, epochs)} :: Simply sets the
learning rate and attributes of a linear regressor so it's ready to be fitted.

\textit{LinearRegressor}\texttt{.fit(X, y)} :: Takes $X$ (numpy.ndarray of shape
($n$, $p$)) and $y$ (numpy.ndarray of shape ($n$)). $n$ is the number of samples
while $p$ is the number of independant variables.

\textit{LinearRegressor}\texttt{.predict(X)} :: takes a matrix of $\mathbf{x}$s,
and spits out their predicted $\textbf{y}$s using the $\beta$ of the last epoch.
Note that the Linear Regressor must be fitted first!

\subsection{Logistic Regressor}

\subsubsection{Description}

The logistic regressor uses the linear regressor but instead of providing a
regressed value for the predict result, it provides a probability of being class
A or class B. It does this by applying the logistic function
$\frac{1}{1+e^{-x}}$ where $x$ is the linear regression result.

\subsubsection{Usage}

\begin{verbatim}
lr = LogisticRegressor(rate, epochs)
lr.fit(X, y)
lr.predict(Z)
\end{verbatim}

\texttt{X} is a numpy.ndarray of shape ($n$, $p$)

\texttt{y} is a numpy.ndarray of shape ($n$)

where $n$ is the number of samples and $p$ is the number of independant
variables

\subsubsection{Properties}

Note: Attributes are meant to be read-only. Modify them at your own risk. I have
not chosen to enforce this since anyone can break it if they wanted anyways.

\textit{LogisticRegressor}\texttt{.rate} :: learning rate

\textit{LogisticRegressor}\texttt{.epochs} :: epochs

\textit{LogisticRegressor}\texttt{.w} :: matrix of values of $\beta$. This is a
numpy.ndarray of shape ($epochs$, $p$).

\textit{LogisticRegressor}\texttt{.errors} :: matrix of values of $\epsilon$.
$\epsilon$ is the error between each predicted $y_i$ and its real value, $d_i$,
such that $\epsilon + y_i = d_i$

\subsubsection{Methods}

\textit{LogisticRegressor}\texttt{.\_\_init\_\_(rate, epochs)} :: Simply sets the
learning rate and attributes of a linear regressor so it's ready to be fitted.

\textit{LogisticRegressor}\texttt{.fit(X, y)} :: Takes $X$ (numpy.ndarray of shape
($n$, $p$)) and $y$ (numpy.ndarray of shape ($n$)). $n$ is the number of samples
while $p$ is the number of independant variables.

\textit{LogisticRegressor}\texttt{.predict(X)} :: takes a matrix of $\mathbf{x}$s,
and spits out their predicted $\textbf{y}$s using the $\beta$ of the last epoch.
Note that the learner must be fitted first.

\subsection{Decision Stump}

\subsubsection{Description}

The decision stump is a decision tree which only has a depth of 1.  It will take
in some samples, each with $n$ features and a label.  The classifier will
determine which feature and value of said feature provides a threshold which
gives the largest confidence gain is classifying the given samples.  While the
samples can have as many labels as wished for, if this is being used as a
standalone classifier a binary labeling is best.

The given dataset comes with an inherit impurity (measured by gini impurity).
The classifier will go through every unique value for each feature given in $X$,
and determine the feature and value that provides the greatest information gain
(the largest decrease in gini impurity).

Uniquely, this classifier will return probabilities of a label given some
features.  For example, for a set of features the classifier may be 10\%
confident that describes label ``A", 70\% confident it describes label ``B", and
10\% confident it's ``C".  This result would look like:
\begin{verbatim}
{ A: 0.1,
  B: 0.7,
  C: 0.1
}
\end{verbatim}

It's also important to note that if a feature's value is numeric, it will try a
$<$ comparison, but if it's not the classifier will try an $=$ comparison.

\subsubsection{Usage}

\begin{verbatim}
ds = DecisionStump()
ds.fit(X, y)
ds.predict(Z)
\end{verbatim}

\subsubsection{Properties}

\textit{DecisionStump}\texttt{.left} :: the probability distribution of the left
side of the decision stump (dict).  These are the probabilities assuming the
value is on the left side, not cumulative.

\textit{DecisionStump}\texttt{.right} :: the probability distribution of the
right side of the decision stump (dict).  These are the probabilities assuming
the value is on the right side, not cumulative.

\textit{DecisionStump}\texttt{.q} :: this is the \texttt{Question} that is asked
by the classifier. It has members \texttt{feature} and \texttt{threshold} and a
method \texttt{match} which takes an array of features and returns \texttt{True}
if it satisfies that threshold or \texttt{False} otherwise.

\subsubsection{Methods}

\textit{DecisionStump}\texttt{.\_\_init\_\_()} :: Just creates the object. There
is nothing needed to be passed in here.

\textit{DecisionStump}\texttt{fit(X, y)} :: Finds the best question to ask to
get the largest about of information gain. It will choose the feature and one of
the values of that feature from \texttt{X}. \texttt{y} is the labelling for each
row in \texttt{X}.

\textit{DecisionStump}\texttt{predict(X)} :: Takes an array of features and
returns the probability distribution for each one in a corresponding array.

\subsection{Threshold}

\subsubsection{Description}

The Threshold leaner will use a decision stump to generate the best threshold to
classify two labels given a set of features. The size of $x$ passed into predict
should be the same size of each feature to ensure correct prediction. Unlike the
deceision stump, predict will return the most likely label rather than the
probability the features correspond to a given label.

\subsubsection{Usage}

\begin{verbatim}
threshold = Threshold()
threshold.fit(X, y)
\end{verbatim}

\subsubsection{Properties}

\textit{Threshold}\texttt{.X} :: the features passed into the fit function

\textit{Threshold}\texttt{.y} :: the labels passed into the fit function,
corresponding to the features.
right side of the decision stump (dict).  These are the probabilities assuming
the value is on the right side, not cumulative.

\subsubsection{Methods}

\textit{Threshold}\texttt{.\_\_init\_\_()} :: Just creates the object. There
is nothing needed to be passed in here.

\textit{Threshold}\texttt{fit(X, y)} :: Uses the decision stump to create a fit
for the threshold. The decision stump actually creates the threshold, but the
threadholding learner actually applies the most likely label.

\textit{Threshold}\texttt{predict(X)} :: Takes a feature set and predicts its
most likely label via a threshold (which is what the decision stump uses)

\end{document}

